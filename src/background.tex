\section{Background}

Indoor localization has been an important topic for years and several techniques have been proposed. In \cite{locSysUbiComp}, the authors describe several localization systems, considering aspects of the technologies in which they are based on, their advantages and limitations. In this section, a brief overview is given of the early indoor localization systems that have influenced and motivated further work on novel techniques.

\subsection{Active Badge}
% Cite paper
Developed at the end of the 90s by the Olivetti Research Laboratory, the system used diffuse infrared technology to locate small badges worn by people. Each badge emitted its own unique identifier periodically or on demand. Fixed sensors distributed in the building collected data from the badges and delivered them to a central computer responsible for aggregating the information and computing badge's position. The main limitations of the system were related to the infrared technology, which suffers with interference from fluorescent light and sunlight. Moreover, multiple beacons were needed in larger rooms, because of the limited range of the technology.

\subsection{Active Bat}
% Cite paper
The system uses ultrasound time-of-flight for determining distances and lateration technique to locate Active Bats carried by people or objects. First, a central controller sends a radio signal to the Bats which respond transmitting an ultrasound signal. Then, sensors mounted in the ceiling calculate distances using time-of-flight of the receive ultrasound signal and forward them to the central controller that computes Bat's position using trilateration or multilateration. The synchronization is done through a reset signal sent from the controller to the ceiling sensors at the same time as the request signal is emitted via radio to the Bats. The system provides accuracy in the order of centimeters in 95\% of the measurements, but presents problems of scalability, high costs and sensitivity to sensors placement due to the ceiling infrastructure that is needed.

\subsection{Cricket}
% Cite paper
Cricket has a decentralized architecture in the sense that each object performs its own triangulation computation. The system is also based on ultrasound and uses radio signals for synchronization of time measurements. Another interesting usage for the radio waves is that they allow the system to discard reflection signals. It is achieved by ignoring any ultrasound heard after the end of the radio frequency signal, once they are considered as echo. Cricket provides positioning information using triangulation, when three or more receivers are available, as well as proximity information when less then three beacons are received.

The decentralized architecture allows higher scalability, less costs with the infrastructure and more privacy. On the other hand, the system is less accurate than Active Bats, for example. Moreover, the computational burden is placed on mobile receivers, which raise, among other aspects, the need for good power management on these devices.

\subsection{RADAR}
% Cite paper
RADAR was developed by Microsoft Research group and it was the first indoor positioning system based on IEEE 802.11 (Wi-Fi) network technology. The base stations measure signal strength and signal-to-noise ratio received from the wireless devices connected to the network and compute their 2D position within the building. A key step of the system implementation is the introduction of an off-line phase, also called training phase, when information about the radio signals are collected and a model for signal propagation is created. Later, during real-time phase (or positioning phase), the previously create model is used for calculate users' location.

RADAR has two possible implementations: one that uses scene analysis and other that is based on lateration techniques. The first one provides better accuracy (3 meters with 50\% of probability) when compared to the second one (4.3 meters with 50\% of probability). However, the scene-analysis approach may require the signal-strength model to be reconstructed in case there are significant changes in the environment.

The main advantage of this system is the possibility of using the same existing infrastructure that provides wireless network for the building, which simplifies the deployment process in terms of costs and complexity besides allowing high scalability.

\subsection{MotionStart magnetic tracker}
% Cite paper
MotionStart was conceived to be used in applications which demand high accuracy level. A fixed transmitting antenna generates DC magnetic-field pulses whose response is measured by the system in three orthogonal axes. These measurements are used to compute the position of the receiving antennas. The known and constant effect of the Earth's magnetic field must be taken into account and isolated in the equations.

The accuracy achieved by this system is very high, on the order of less than 1 millimeter. However, metallic materials in the environment influence the measurements and tend to compromise this level of accuracy. Moreover, the high costs of this solution frequently make it unfeasible in many applications.

\subsection{Easy Living}
% Cite paper
Also developed by Microsoft Research group, EasyLiving is one of the first systems that use computer vision technology to provide positioning capability. Further research on this technique % “Integrated Person Tracking Using Stereo, Color, and Pattern Detection,” 
demonstrated that multimodal processing, like people and objects recognition, may significantly enhance accuracy, but it also increases computing burden. By the time the system was designated, the amount of computing power required for processing data from real-time 3D cameras exceeded the capabilities of most devices. Nevertheless, nowadays, the use of accelerators as GPUs and FPGAs can overcome this limitation.