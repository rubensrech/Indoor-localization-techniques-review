\section{Background}

Indoor localization has been an important topic for years and several techniques have been proposed. In \cite{locSysUbiComp}, the authors describe several localization systems, considering aspects of the technologies in which they are based on, their advantages and limitations. In this section, some important concepts are described and a brief overview is given of the early indoor localization systems that have influenced and motivated further work on novel techniques.

\subsection{Active Badge}
Developed at the end of the 90s by the Olivetti Research Laboratory \cite{activeBadge}, the system used diffuse infrared technology to locate small badges worn by people. Each badge emitted its own unique identifier periodically or on demand. Fixed sensors distributed in the building collected data from the badges and delivered them to a central computer responsible for aggregating the information and computing badge's position. The main limitations of the system were related to the infrared technology, which suffers with interference from fluorescent light and sunlight. Moreover, multiple beacons were needed in larger rooms, because of the limited range of the technology.

\subsection{Active Bat}
The system uses ultrasound time-of-flight for determining distances and lateration technique to locate Active Bats carried by people or objects. First, a central controller sends a radio signal to the Bats which respond transmitting an ultrasound signal. Then, sensors mounted in the ceiling calculate distances using time-of-flight of the receive ultrasound signal and forward them to the central controller that computes Bat's position using trilateration or multilateration. The synchronization is done through a reset signal sent from the controller to the ceiling sensors at the same time as the request signal is emitted via radio to the Bats. The system provides accuracy in the order of centimeters in 95\% of the measurements, but presents problems of scalability, high costs and sensitivity to sensors placement due to the ceiling infrastructure that is needed.

\subsection{Cricket}
Cricket \cite{cricket} has a decentralized architecture in the sense that each object performs its own triangulation computation. The system is also based on ultrasound and uses radio signals for synchronization of time measurements. Another interesting usage for the radio waves is that they allow the system to discard reflection signals. It is achieved by ignoring any ultrasound heard after the end of the radio frequency signal, once they are considered as echo. Cricket provides positioning information using triangulation, when three or more receivers are available, as well as proximity information when less then three beacons are received.

The decentralized architecture allows higher scalability, less costs with the infrastructure and more privacy. On the other hand, the system is less accurate than Active Bats, for example. Moreover, the computational burden is placed on mobile receivers, which raise, among other aspects, the need for good power management on these devices.

\subsection{RADAR}
RADAR \cite{RADAR} was developed by Microsoft Research group and it was the first indoor positioning system based on IEEE 802.11 (Wi-Fi) network technology. The base stations measure signal strength and signal-to-noise ratio received from the wireless devices connected to the network and compute their 2D position within the building. A key step of the system implementation is the introduction of an off-line phase, also called training phase, when information about the radio signals are collected and a model for signal propagation is created. Later, during real-time phase (or positioning phase), the previously create model is used for calculate users' location.

RADAR has two possible implementations: one that uses scene analysis and other that is based on lateration techniques. The first one provides better accuracy (3 meters with 50\% of probability) when compared to the second one (4.3 meters with 50\% of probability). However, the scene-analysis approach may require the signal-strength model to be reconstructed in case there are significant changes in the environment.

The main advantage of this system is the possibility of using the same existing infrastructure that provides wireless network for the building, which simplifies the deployment process in terms of costs and complexity besides allowing high scalability.

\subsection{MotionStart magnetic tracker}
MotionStart \cite{motionStart} was conceived to be used in applications which demand high accuracy level. A fixed transmitting antenna generates DC magnetic-field pulses whose response is measured by the system in three orthogonal axes. These measurements are used to compute the position of the receiving antennas. The known and constant effect of the Earth's magnetic field must be taken into account and isolated in the equations.

The accuracy achieved by this system is very high, on the order of less than 1 millimeter. However, metallic materials in the environment influence the measurements and tend to compromise this level of accuracy. Moreover, the high costs of this solution frequently make it unfeasible in many applications.

\subsection{Easy Living}
Also developed by Microsoft Research group, EasyLiving \cite{easyLiving} is one of the first systems that use computer vision technology to provide positioning capability. Further research on this technique demonstrated that multimodal processing, like people and objects recognition, may significantly enhance accuracy, but it also increases computing burden. By the time the system was designated, the amount of computing power required for processing data from real-time 3D cameras exceeded the capabilities of most devices. Nevertheless, nowadays, the use of accelerators as GPUs and FPGAs can overcome this limitation.

\subsection{WiFi} 
WiFi is one of the major technologies used in many indoor localization systems because of it's availability in indoor environments and support of IEEE 802.11 WiFi standard on most mobile devices. Both Propagation and Fingerprinting techniques based on WiFi have been proposed. Propagation based techniques aim to model the relation between the received signal and distance without site surveying. Although these techniques can be easily deployed without any prior calibration, they fail to perform well on heterogeneous phones and they comparatively have lower accuracy than Fingerprinting techniques. Fingerprinting techniques leverage the recorded WiFi Access Points (APs) signatures, called fingerprints, to estimate the device location. 

Typical Fingerprinting localization technique based on WiFi works in two phases - Offline Phase and Tracking Phase. In offline phase, the received signal strength (RSS) readings from multiple APs are recorded at a known location. And in tracking phase, RSS measurements at an unknown location are matched against the stored fingerprints to estimate the best location match either deterministically or probabilistically \cite{wideep}. However, the fingerprinting techniques suffer from noise and uncertainty which in turn reduces the accuracy of the determined location. Dead reckoning is another technique which can be combined with fingerprinting which improves the accuracy by a certain percentage but dead reckoning is known to suffer from error accumulation over time \cite{cnn}.

\subsection{Radio Frequency}
The advancement in Internet-of-Things (IoT) has made remarkable changes in the field of Telecommunications. Such IoT systems comprise of multiple smart sensors and are usually coupled with the mobile devices, which in recent times already contains quite a lot of smart sensors. Localization is an unavoidable requirement for efficient deployment of such sensors in an indoor environment. In such scenarios, localization is done according to the radio frequency (RF) propagation channel. Multiple RF features used mainly for localization include received signal strength (RSS), channel transfer function (CTF), and frequency coherence function (FCF) \cite{cascade}.

CTF represents the superposition of the complex-valued gain associated with the multipath components of the wireless channel. CTF is an RF signature that would be distinctively unique for every spatial position in the physical environment. Under frequency selective fading, this RF signatures becomes more sensitive to channel variations due to the rapid fluctuations of the complex-valued gains associated with its multipath components. FCF is another RF channel metric that represents the frequency domain coherence of the radio channel and is known for its slow changing nature in the spatial domain which makes it a strong candidate as in RF signature \cite{real-dl}.

Several localization techniques based on these RF features have been proposed in the literature. Most of them still suffer from problems, such as multipath effect and no line-of-sight path, arising due to the complexity of the indoor wireless channel.

\subsection{Machine Learning/Deep Learning}
Machine learning is a field of artificial intelligence dealing with algorithms that improve performance over time with experience. The measure of performance is how well the algorithm predicts the output value given a set of variables or attributes. Machine learning algorithms provide excellent solutions for building models that generalize well given large amounts of data with many attributes by discovering patterns and trends in the data. With an increasing number of sensors being made available in the majority of mobile devices, large amounts of data can be collected and used to aid in the localization process. Machine learning algorithms are a natural solution for sifting through these large datasets and determining the important pieces of information for localization, building accurate models to predict an indoor position \cite{ml}.

Deep learning architectures have the ability to learn automatically features with higher levels of abstractions as well as complex functions that map the input to the output. They are composed of multiple levels of non-linear operations which are inspired by human brains \cite{dl}. While traditional ML techniques work well at approximating simpler input-output functions, computationally intensive deep learning models are capable of dealing with more complex input-output mappings and can deliver better accuracy. Furthermore, with the increase in the available computational power on mobile devices, it is now possible to deploy deep learning techniques on smartphones \cite{cnn}.