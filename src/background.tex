\section{Background}

Indoor localization has been an important topic for years and several techniques have been proposed. In \cite{locSysUbiComp}, the authors describe several localization systems, considering aspects of the technologies in which they are based on, their advantages and limitations. In this section, a brief overview of the early indoor localization systems is given.

\subsection{Active Badge}
Developed at the end of the 90s by the Olivetti Research Laboratory, the system used diffuse infrared technology to locate small badges worn by people. Each badge emitted its own unique identifier periodically or on demand. Fixed sensors distributed in the building collected data from the badges and delivered them to a central computer responsible for aggregating the information and computing badge's position. The main limitations of the system were related to the infrared technology, which suffers with interference from fluorescent light and sunlight. Moreover, multiple beacons were needed in larger rooms, because of the limited range of the technology.

\subsection{Active Bat}
The system uses ultrasound time-of-flight for determining distances and lateration technique to locate Active Bats carried by people or objects. First, a central controller sends a radio signal to the Bats which respond transmitting an ultrasound signal. Then, sensors mounted in the ceiling calculate distances using time-of-flight of the receive ultrasound signal and forward them to the central controller that computes Bat's position using trilateration or multilateration. Synchronization is done through a reset signal sent from the controller to the ceiling sensors at the same time as the request signal is emitted via radio to the Bats. The system provides accuracy in the order of centimeters in 95\% of the measurements, but presents problems of scalability, high costs and sensitivity to sensors placement due to the ceiling infrastructure that is needed.

\subsection{Cricket}

\subsection{RADAR}

\subsection{MotionStart magnetic tracker}

\subsection{Easy Living}
